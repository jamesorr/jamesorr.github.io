Scientists that have reported uncertainties have coded it themselves
using either Monte Carlo or Gaussian approaches. For the latter, they
have had to implement equations such as

$$
      \sigma_{\textrm{pCO}_2}^2 =
                    \left(\frac{\partial \textrm{pCO}_2}{\partial A_{\textrm{T}}}\right)^2 \sigma_{A_{\textrm{T}}}^2 
                  + \left(\frac{\partial \textrm{pCO}_2}{\partial C_{\textrm{T}}}\right)^2 \sigma_{C_{\textrm{T}}}^2
                  + \left(\frac{\partial \textrm{pCO}_2}{\partial T}\right)^2 \sigma_T^2
                  + \left(\frac{\partial \textrm{pCO}_2}{\partial S}\right)^2 \sigma_S^2
                  + \left(\frac{\partial \textrm{pCO}_2}{\partial K_1}\right)^2 \sigma_{K_1}^2
                  + \left(\frac{\partial \textrm{pCO}_2}{\partial K_2}\right)^2 \sigma_{K_2}^2
		  + \, \ldots
		  \textrm{,}
$$

while specifying the uncertainties $$ \textrm{(}\sigma\textrm{'s)} $$
and calculating the sensitivities (partial derivatives) which vary
regionally and with depth.  Furthermore, the few laboratories that
implemented uncertainty propagation have not done so consistently.
